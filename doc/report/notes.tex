
%\documentclass[twocolumn]{article}
\documentclass{article}

\usepackage[utf8]{inputenc}
\usepackage{algorithm}
\usepackage{algorithmic}
\usepackage{mydefs}
\usepackage{proof}
\usepackage{amsmath}
\usepackage{amsthm}
\usepackage{amsfonts}
\usepackage{graphicx}
\usepackage{figlatex}
\usepackage{enumerate}
\usepackage[a4paper]{geometry}
%\usepackage[right=3.4cm,left=3.4cm]{geometry}

\newtheorem{definition}{Definition}
\newtheorem{proposition}{Proposition}
\newtheorem{lemma}{Lemma}
\newtheorem{theorem}{Theorem}
\newtheorem{remark}{Remark}

\newcommand{\event}[1]{\ensuremath{\mathsf{event}(#1)}}
\newcommand{\hist}[1]{\ensuremath{\mathsf{hist}(#1)}}
\newcommand{\origin}[1]{\ensuremath{\mathsf{origin}(#1)}}
\newcommand{\conf}[1]{\ensuremath{\mathsf{Conf}(#1)}}
\newcommand{\conds}[1]{\ensuremath{\mathsf{Conds}(#1)}}
\newcommand{\events}[1]{\ensuremath{\mathsf{Events}(#1)}}
\newcommand{\confl}{\ensuremath{\mathord{\#}}}
%\newcommand{\conc}[2]{\ensuremath{#1 \parallel #2}}

\newcommand{\h}{\ensuremath{\mathcal{H}}}
\newcommand{\pe}{{\sc PE}}
\newcommand{\evolves}{{\sqsubseteq}}
\newcommand{\peupdate}{{\sc PE-Update}}
\newcommand{\peupdatecont}{{\sc PE-Update-Context}}
\newcommand{\peupdatepostexisting}{{\sc PE-Update-Postset-Existing}}
\newcommand{\peupdatepostnew}{{\sc PE-Update-Postset-New}}
\newcommand{\pehistdup}{{\sc PE-History-Duplicated}}
\newcommand{\peeventdup}{{\sc PE-Event-Duplicated}}

\title{\pe{} update procedure}
\author{César Rodríguez}
%\date{May 2010}

\begin{document}
\maketitle

% Introduccion de dos parrafos sobre que es un desenrollado
% Que es una red contextual
% Formas de construir el desenrollado de la red contextual
% Motivacion para construir un desenrollado contextual: es mas pequeño
% Retos: la computacion de conc(S) es mas compleja, hay cambios fundamentales
% en el algoritmo de computo; no tiene mucho sentido reutilizar el codigo de la
% herramienta vieja

\section{Introduction}
\section{Contextual Petri nets}

\begin{definition} A \emph{contextual Petri net} is a tuple $N = (P, T, F, C,
m_0)$ \FIXME{...}
\end{definition}

In this work we consider contextual 1-safe nets.  $N$ is finite, $P$ and $T$ is
finite :)

\section{Occurrence nets and contextual unfoldings}

\begin{definition} A \emph{contextual occurrence net} is a 1-safe contextual
Petri net $N = (P, T, F, C, m_0)$ \FIXME{...}
\end{definition}

\begin{figure}
%\centering{\includegraphics{fig/FIXME.fig}
\FIXME{Fig. 4 in McMillan's Complete}
\caption{Inductive definition of the full unfolding $\unf{N}$}
\label{fig:inductive.definition}
\end{figure}

\begin{definition} The \emph{full unfolding} of a contextual \FIXME{bounded,
check definitions} Petri net $N$, denoted $\unf{N}$, is the contextual
occurrence net $(P', T', F', C', m'_0)$ defined by the induction rules of
\rfig{inductive.definition}. \FIXME{}
\end{definition}

\begin{proposition} The full unfolding $\unf{N}$ of a contextual net $N$ is
complete. \FIXME{define completeness}
\end{proposition}

\section{A complete and finite prefix}

\begin{definition} An \emph{enriched occurrence net} is a pair $E = (N, \chi)$
where $N = (P, T, F, C, m_0)$ is an occurrence net and $\chi : T \to X$
\FIXME{...}.  An enriched occurrence net is called \emph{closed} if
\FIXME{...}.
\end{definition}

\FIXME{Define here configuration of an enriched occurrence net}

\begin{definition} An \emph{enriched event} is a pair $(e, H_e)$, where $e$ is
\FIXME{...}.
\end{definition}

\begin{definition} An \emph{enriched prefix} of the unfolding $\unf{N}$ is any
\emph{closed} enriched occurrence net $E$ such that $N_E$ \FIXME{...}.
\end{definition}

\begin{definition} Given two enriched prefixes, $E_1 = (N_1, \chi_1)$ and $E_2
= (N_2, \chi_2)$, we say that $E_1$ is \emph{a prefix} of $E_2$, and write $E_1
\unlhd E_2$, iff $N_1$ is a prefix of $N_2$ and for all $e \in T_1$ we have
$\chi_1 (e) \subseteq \chi_2 (e)$.
\end{definition}

%\begin{lemma} Complete lattice.  \FIXME{We really need it?}
%\end{lemma}

\FIXME{Define the order $C_1 \sqsubset C_2$}

\subsection{Adequate orders}

\begin{definition} \label{def:adequate} A partial order $\prec$ on the finite
configurations of the full unfolding is called \emph{adequate} if
\begin{itemize} \item $\prec$ is well founded, \item $C_1 \sqsubset C_2$
implies $C_1 \prec C_2$, and \item $\prec$ is preserved by finite extensions,
that is,  if $C_1 \prec C_2$, and $\marking{C_1} = \marking{C_2}$, and $C_1
\sqsubset C_1 \cup E$ for some extension $E$, and $C_2 \sqsubset C_2 \cup E'$
for some extension $E'$ isomorphic to $E$, then $C_1 \cup E \prec C_2 \cup E'$.
\end{itemize}
\end{definition}

This definition is more general than the one in \FIXME{make, of course, some
clarifications here} ERVpaper, as every adequate order according to that
definition is also an adequate order according to this definition, but not in
the other way round.

From now on, we fix a partial order $\prec$ on the finite configurations
$\unf{N}$ for some n-bounded net $N$.

\begin{definition} An enriched event $(E, H_e)$ of the full unfolding $\unf{N}$
is called \emph{cut-off} if either $\marking{H_e} = m$ with $m$ the initial
marking of $N$ or there exists another enriched event $(e', H_{e'})$ verifying
\begin{itemize} \item $\marking{H_e} = \marking{H_{e'}}$, and \item $H_e' \prec
H_e$ \end{itemize}
\end{definition}

\begin{definition} The \emph{truncation} $\trunc{N}$ of the full unfolding is
an enriched occurrence net defined as the greatest enriched prefix (w.r.t to
the prefix ordering $\unlhd$) of the full unfolding which does not contain
cut-offs.
\end{definition}

\begin{theorem} $\trunc{N}$ is finite.
\end{theorem}

\begin{proof}
We want to prove that the number of enriched events in $\trunc{N}$ is finite.
By contradiction, assume this number is infinite. As any history of an enriched
event is a configuration, and, in particular, a \emph{finite} configuration, we
also know that the number of finite configurations of $\trunc{N}$ is infinite.
In the sequel, we see that under these assumptions we can use K\"{o}nig's lemma
to find an infinite sequence of configurations $C_1, C_2, \ldots$ with $C_i
\sqsubset C_{i+1}$ and $\marking{C_1} = \marking{C_i}$ for $i \ge 1$, yielding
to the fact that $C_2$ is a cutoff, which is a contradiction to the definition
of $\trunc{N}$.

We define a graph $G = (V, E)$ whose set of vertices $V$ is the set of finite
configurations of $\trunc{N}$ plus the \textit{empty configuration}
$\emptyset$, and whose set of edges is a subset of the computational order
relation $\sqsubseteq$:

\begin{itemize}
\item $V = \conf{\trunc{N}} \cup \{\emptyset\}$ \FIXME{only finite ones!}
\item $E = \{(C_1, C_2) \in V \times V \mid C_1 \sqsubseteq C_2 \text{ and }
|C_1| + 1 = |C_2|\}$
\end{itemize}

We write $C_1 \to C_2$ for $(C_1, C_2) \in E$.  We now prove three properties
of $G$:

\begin{description}
\item[] \emph{G has finite degree.}  We show that for any $C_1$ there is only a
finite number of configurations $C_2$ such that $C_1 \to C_2$.

Consider any pair of configurations $C_1$, $C_2$ with $C_1 \to C_2$.  Then $C_1
\sqsubseteq C_2$ and $|C_1| + 1 = |C_2|$.  By definition, $C_1 \sqsubseteq C_2$
implies $C_1 \subseteq C_2$.  Then we can write $C_2 = C_1 \cup \{e\}$ for some
$e \notin C_1$.  From the fact that $C_1 \sqsubseteq C_2$ we also know that
$\lnot (e_2 \nearrow e_1)$ for all $e_1 \in C_1$ and all $e_2 \in C_2 \setminus
C_1$.  In particular $\lnot (e \nearrow e_1)$ for all $e_1 \in C_1$, so $e$ can
fire after any event $e_1$ of $C_1$.  This, together with the fact that $C_2$
is a configuration, implies that $\pre{e} \subseteq \cut{C_1}$. From this, it
follows that $e \in \post{\cut{C_1}}$.

Now notice that $\post{\cut{C_1}}$ is finite, because $C_1$ is finite (by
definition) and the preset and postset of each event in $\trunc{N}$ is finite.
So there is, at most, a finite number of \emph{different} events $e \in
\post{\cut{C_1}}$ and hence a finite number of different sets $C_1 \cup \{e\}$.
As we saw previously, any $C_2$ such that $C_1 \to C_2$ is one of these $C_1
\cup \{e\}$ for $e \in \post{\cut{C_1}}$.  Hence, for a given $C_1$, there is
at most a finite number of different edges $C_1 \to C_2$ in $G$.

\item[] \emph{G is connected.}  We show by induction on the size of a
configuration $C$ that there exists a path from the empty configuration
$\emptyset$ to $C$.  In the base case, $|C| = 0$ and consequently $C =
\emptyset$.  The statement is trivially true.  For the step case, assume that
we can find a path for configurations of up to size $n$, or more formally, that
$\emptyset \to^* C$ if $|C| \le n$ and let $C'$ be a configuration of size
$n+1$.  From $C'$ we can extract a event $e$ maximal w.r.t. the relation
$\nearrow$.  This is always possible because $C'$ is finite and there is no
loops in $\nearrow$ restricted to $C'$.  As there exists no other $e' \in C'$
with $e \nearrow e'$, we can assure that $C' \setminus \{e\}$ is a
configuration.  Furthermore, it is a configuration of size $n$ for which the
induction hypothesis applies.  Finally, notice that $C' \setminus \{e\}
\sqsubseteq C'$, because $C' \setminus \{e\} \subseteq C'$ and the maximality
of $e$.  Consequently, $C' \setminus \{e\} \to C'$.

\item[] \emph{$G$ has an infinite number of vertices}.  By hypothesis, we know
that there is an infinite number of finite configurations in $\trunc{N}$.  Each
finite configuration is a vertex in $G$.
\end{description}

In the light of these properties, K\"{o}nig's lemma guarantees the existence in
$G$ of an infinite sequence of configurations $(C_i)_{i \ge 1}$ such that $C_1
= \emptyset$ and $C_i \to C_{i+1}$.  By hypothesis, $N$ is n-bounded, so the
number of reachable markings in $N$ is finite.  As each configuration $C_i$ has
an associated marking $\marking{C_i}$, this implies that from $(C_i)_{i \ge 1}$
we can extract a infinite subsequence of configurations $C_{i_1}, C_{i_2},
\ldots$ marked with the same marking, that is, verifying:

\begin{itemize}
\item (subsequence) $i_j < i_{j+1}$ for $j \ge 1$, and
\item (same marking) $\marking{C_{i_1}} = \marking{C_{i_j}}$ for $j \ge 2$, and
\end{itemize}

Consider now the configurations $C_{i_1}$ and $C_{i_2}$.  By construction, we
know that $C_{i_1} \to^* C_{i_2}$.  This, together with the fact that relation
$\sqsubseteq$ is transitive, implies that $C_{i_1} \sqsubseteq C_{i_2}$ and by
\rdef{adequate}, that $C_{i_1} \prec C_{i_2}$.  But, by construction, we know
that $\marking{C_{i_1}} = \marking{C_{i_2}}$, so $C_{i_2}$ is a cutoff.  This
is a contradiction to the fact that $\trunc{N}$ is free of cutoffs.

Hence, the number of enriched events in $\trunc{N}$ is finite.
\end{proof}

%\begin{proof} By contradiction, assume that $\trunc{N}$ is infinite. We can
%show that there is an infinite computation in $\trunc{N}$ (Konig's lemma and
%finite preset and postset of every event).  In this computation, consider the
%infinite sequence of markings reached after the firing of each event. As there
%is a finite number of markings, one marking is repeated.  Consider now the
%first repetition.  That enriched event is a cutoff, and it is present in the
%net, because $\trunc{N}$ is closed.  This is a contradiction.
%\end{proof}

\FIXME{Move this definition and the next proposition to other place}
\begin{definition}  Given a n-bounded net $N$, an enriched prefix $E = (N_E,
\chi_E)$ of its full unfolding is called \emph{complete} if for every reachable
marking $m$ in $N$ we can find a configuration $C$ in $E$ such that
$\marking{C} = m$.
\end{definition}

\begin{proposition} \label{pro:unfolding.complete} Let $N$ be a contextual net.
Then, the full unfolding $\unf{N}$ is complete.
\end{proposition} 

\begin{lemma}
\label{lem:cutoff.elimination}
Consider the full unfolding $\unf{N}$ of a n-bounded contextual net $N$.  If a
finite configuration $C \in \conf{\unf{N}}$ contains at least one cutoff, then
there exists another configuration $C'$ such that $\marking{C} = \marking{C'}$
and $C' \prec C$.
\end{lemma}

\begin{proof}
We say that a configuration $C$ \emph{contains} a cutoff if there exists an
event $e \in C$ such that the enriched event $(e, C\sem{e})$ is a cutoff.
Notice that, as the full unfolding is a closed prefix and $C$ is a
configuration, the enriched event $(e, C\sem{e})$ is always contained in it.

Assume that $C$ contains the cutoff $(e, H_e)$, for some $e \in C$.  Then, by
definition, we know that either $\marking{H_e} = m$ with $m$ the initial
marking of $N$ or there exists another enriched event $(e', H_{e'})$ such that
$\marking{H_e} = \marking{H_{e'}}$ and $H_{e'} \prec H_e$.  In both cases,
we regard the history $H$ defined as $H \bydef{=} \emptyset$ in the first case
and $H \bydef{=} H_{e'}$ in the second case.  Notice that in any case, $H \prec
H_e$ (if $H = \emptyset$, then $H \sqsubset H_e$ and, by definition, $H \prec
H_e$).

As $\marking{H_e} = \marking{H}$, it is possible to find an extension of $H$
firing exactly the same transitions as those labeling the set $C \setminus
H_e$.  In other words, it must be possible to find an extension $E$ of $H$
isomorphic to $C \setminus H_e$, such that $\marking{H_e \cup (C \setminus
H_e)} = \marking{H \cup E}$.  This follows from the fact that $\unf{N}$ unfolds
\emph{as much as possible} and can be proved by induction on the size of the
set $C \setminus H_e$:

\begin{itemize}
\item{Base}.  $|C \setminus H_e| = 1$.  Assume that $C \setminus H_e =
\{e'\}$.  As $\marking{H_e} = \marking{H}$, then the transition $t$ labeling
$e'$ is enabled at $\marking{H}$ and, by the definition of full unfolding,
there exists an event $e''$ labeled by $t$ such that $H \cup \{e''\}$ is a
configuration.  Hence, $\{e''\}$ is an extension of $H$ isomorphic to $C
\setminus H_e$.

\item{Step}.  Assume we can find extensions of $H$ isomorphic to $C \setminus
H_e$ up to size $n$.  We show that it is also possible for size $n+1$.  Assume
that $|C \setminus H_e| = n + 1$ and consider any maximal event $e'$ of $C
\setminus H_e$ w.r.t. the relation $\nearrow$ (it must exists because
$\nearrow$ is acyclic in $C$).  As $e'$ can fire after any event in $C
\setminus H_e$, and therefore any event in $C$, we know that $C \setminus
\{e'\}$ is a configuration and that the set $(C \setminus H_e) \setminus
\{e'\}$ is an extension of $H_e$ of size $n$.  Then, the induction hypothesis
applies and says that we can find an extension $E$ of $H$, isomorphic to $(C
\setminus H_e) \setminus \{e'\}$ such that $\marking{C \setminus \{e'\}} =
\marking{H \cup E}$.  But now notice that the transition $t$ labeling $e'$ is
enabled at $\marking{H \cup E}$ and, hence, there exists an event $e''$ labeled
by $t$ such that $H \cup E \cup \{e''\}$ is a configuration and $E \cup
\{e''\}$ is an extension of $H$ isomorphic to $C \setminus H_e$.  \end{itemize}

We claim now that $C' \bydef{=} H \cup E$ is the configuration we are searching
for.  To see this, we just have to verify that $C' \prec C$.  We do it using
the third condition of \rdef{adequate}.  Indeed, $H_e$ and $H$ are
configurations with the same marking; by hypothesis $H_e \sqsubset C$ and by
construction $H \sqsubset H \cup E$ and $E$ is isomorphic to $C \setminus H_e$.
Hence $H \cup E \prec C$.
\end{proof}

\begin{theorem} Let $N$ be a n-bounded net.  Then, $\trunc{N}$ is complete.
\end{theorem}

\begin{proof}
We know after \rpro{unfolding.complete} that the full unfolding of $N$ is a
complete enriched prefix.  Therefore, for any marking $M$ of $N$ we can find a
configuration $C \in \conf{\unf{N}}$ with $\marking{C} = M$.  We show now that
for such configuration $C$ we can find another configuration $C'$ with the same
marking and free of cutoffs that, consequently, belongs to $\trunc{N}$.

Let $C \in \conf{\unf{N}}$ be a configuration of the full unfolding.  If $C$ is
free of cutoffs, then all enriched events $(e, C\sem{e})$ are enriched events
of $\trunc{N}$ and therefore $C$ is a configuration of $\trunc{N}$.

So, let us assume that $C$ contains a cutoff. By \rlem{cutoff.elimination} we
can find another configuration $C_1 \in \conf{\unf{N}}$ with $\marking{C} =
\marking{C_1}$ and $C_1 \prec C$.  If $C_1$ still contains cutoffs, we can
apply again \rlem{cutoff.elimination} and find another configuration $C_2 \prec
C_1$ with $\marking{C_2} = \marking{C}$.  Furthermore, as $\prec$ is well
founded, the number of times we have to apply \rlem{cutoff.elimination} in
order to find a configuration $C_n$ free of cutoffs is finite,  otherwise we
would either have an infinite decreasing sequence of configurations w.r.t. the
order $\prec$, or a counterexample for \rlem{cutoff.elimination}.
\end{proof}


\newpage
\section{Update procedure for the \pe{} set}

We consider a contextual 1-safe Petri net $N = (P, T, F, C, m_0)$ (definition
omitted; $F \subseteq P \times T \cup P \times T$ is the \emph{flow relation}
while $C \subseteq P \times T$ is the \emph{context relation}) and its
\emph{full} unfolding $\unf{N} = (P', T', F', C', m_0')$.  Conditions in $P'$
are labeled with places in $P$, and events in $T'$ are labeled with transitions
in $T$.  We write $y = \origin{n}$ if place (resp. transition) $y$ labels
condition (resp. event) $n$.  As usual, we write $\pre{x}$, $\post{x}$ and
$\cont{x}$ for the \emph{preset}, \emph{postset} and \emph{context},
respectively, of some place, transition, condition or event $x \in P \cup T
\cup P' \cup T'$.

\paragraph{History graph} Let $e \in T'$ be an event of $\unf{N}$.  A
\emph{history of $e$} is a set $H_e \subseteq T'$ verifying some particular
constraints (definition omitted).  For any history $H_e$ we have that $e \in
H_e$, and we write $\event{H_e}$ to to denote the maximal event of $H_e$ w.r.t.
the \emph{asymmetric conflict} relation $\nearrow$ (it is easy to see that this
event is always $e$).  We extend the definition of $\event$ to sets of
histories in the natural way.  We denote by $\hist{e}$ the set of histories of
$e$ \emph{currently present in the computed prefix of the unfolding}.  The
definition of $\hist{e}$ is also extended to subsets of $T'$.

Storage of the histories associated to every event is done thanks to a directed
graph, so called the \emph{history graph}.  Each node of the graph (indirectly)
represents one history.  Nodes are labeled with events, in such a way that a
node representing some history of $e$ is labeled with $e$.  Different graph
nodes can be labeled with the same event, provided they represent different
histories for that event.

\begin{definition}

A \emph{history graph} is a graph $\h = (V, E)$ whose set of vertices $V$
coincides with the set of histories present in the unfolding, formally $V =
\hist{T'}$, and whose set of edges $E \subseteq V \times V$ is the smallest set
verifying the next inference rule:

\begin{center}
\infer{(H_e, H) \in E$ for any $H \in P \cup C}
	{P, C \subseteq V &
	\event{P} = \pre{\pre{e}} \cup \pre{\cont{e}} &
	\event{C} \subseteq \cont{\pre{e}} &
	\forall H_{e'} H_{e''} \in P \cup C, \ e' \neq e'' \land \neg (H_{e'} \# H_{e''})
	}
\end{center}

\end{definition}

%Let $H_e \in V$ be some node representing a history for an event $e$.  Let $P
%\subseteq V$ and $C \subseteq V$ be two sets of nodes such that (1) $\event{P}
%= \pre{\pre{e}} \cup \pre{\cont{e}}$ and $\event{C} \subseteq \cont{\pre{e}}$,
%and (2) for each pair of histories $H_{e'}$, $H_{e''} \in P \cup C$ we have $e'
%\neq e''$ and $\neg (H_{e'} \# H_{e''})$.  Then $(H_e, H) \in E$ for any $H \in P \cup
%C$.

\paragraph{Conflict relation between histories}  We define the conflict
relation between histories as $H_1 \# H_2 $ iff
\begin{itemize}
\item $H_1$ or $H_2$ is a cutoff, or 
\item $H_1$ or $H_2$ is in \pe{}, or
\item there exists $e_1 \in H_1$ and $e_2 \in H_2 \setminus H_1$ with $e_2
\nearrow e_1$
\item there exists $e_2 \in H_2$ and $e_1 \in H_1 \setminus H_2$ with $e_1
\nearrow e_2$
\end{itemize}

\paragraph{Procedure \peupdate{}}  The \pe{} set is updated with new histories
after the addition of each new history $H_e$ to $\unf{N}$.  \ralg{pe.update} is
currently used for this task.  This procedure receives as input a new history
$H_e$ and generates all new histories for events $e'$ such that $e \nearrow
e'$.  Note that the existence of a new history $H_e$ may trigger the creation
of new histories for every event $e''$ such that $e \nearrow^* e''$.  Our
procedure operates step by step, adding new histories just for events $e'$ with
$e \nearrow e'$.  If new histories appear for $e'$, then the procedure will be
executed again, recursively triggering the discovery of all new histories
appearing due to the addition of $H_e$.

Procedure \peupdate{} operates in three steps.  First, we generate histories
for events $e'$ such that $\cont{e} \cap \pre{e'} \neq \emptyset$ and $\post{e}
\cap \pre{e'} = \emptyset$.  Clearly, in this situation, $e \nearrow e'$.  Let
$\hist{e'} = \{H_1, \ldots, H_n\}$ be the set of histories associated to $e'$
when calling \ralg{pe.update} for $H_e$.  It is easy to see that all histories
that one can build for $e'$ using the new history $H_e$ are in the set $\{H_1
\cup H_e, \ldots, H_n \cup H_e\}$.  Procedure \peupdatecont{}, in
\ralg{pe.update.cont}, filters this set and updates \pe{} with only consistent
histories from it.  Briefly, there is a new history $H$ for event $e'$ if
$H_i$, for $1 \le i \le n$, has edges in the graph $\h$ to histories $H_{i,1},
\ldots, H_{i,m}$, and $\lnot H \# H_{i,j}$, for $1 \le j \le m$, and $H$ doesn't
consume conditions in the preset or context of $e'$.  If this is the case, then
the history $H \cup H_i$ is new for $e'$.

It can also be the case that the history $H \cup H_i$ already exists in the
unfolding \FIXME{search for an example}.  In this case, we just discard the
history.  We use procedure \pehistdup{} to know when a history has already been
computed.

Procedure \peupdatepostexisting{} (\ralg{pe.update.post.existing}) searches for
new histories for events $e'$ such that $\post{e} \cap (\pre{e'} \cup
\cont{e'}) \neq \emptyset$ and $e'$ is already present in the current state of
the unfolding.  Such an event $e'$ has a new history if it is possible to find
a combination of histories for the events generating its preset and context
that is free of conflicts and includes $H_e$.  More formally, let $\{c_1,
\ldots, c_n\} = \pre{e'} \cup \cont{e'}$ be the set of conditions defining the
preset and context of $e'$.  Now assume that $\{e_1, \ldots, e_m\} =
\pre{\{c_1, \ldots, c_n\}}$ are the events generating $\{c_1, \ldots, c_n\}$.
Finally, wlog, assume that $e = e_1 = \pre{c_1}$.  Notice that $H_e$ could also
generate other conditions $c_i$ different from $c_1$.  Then, there exists a new
history for $e'$ if there exist histories $H_{e_1}, \ldots, H_{e_m}$ for events
$e_1, \ldots, e_m$ such that $H_{e_1} = H_e$ and for $1 \le i \le m$ and $1 \le
j < i$ we have that

\begin{itemize}
\item $H_{e_i}$ does not consume any condition from $c_1, \ldots, c_n$, and
\item $\lnot H_{e_j} \# H_{e_i}$
\end{itemize}

In that situation, $e'$ has a new history consisting on the union of all
$H_{e_i}$ for $1 \le i \le m$ and $\{e'\}$.  Notice that it is possible to
generate several times the same history \FIXME{search for an example}.  For
this reason, we use procedure \pehistdup{} to determine if this history already
exists in the current state of the unfolding, or it is really new.  We only
append it to the unfolding in the latter case.

Finally, procedure \peupdatepostnew{} (\ralg{pe.update.post.new}) tries to
append to \pe{} histories for events still not present neither in the unfolding
nor \pe{}.  For that, it explores the possibility of adding events $e'$ reading
or consuming conditions in the post-set of $e$, formally $\post{e} \cap
(\pre{e'} \cup \cont{e'}) \neq \emptyset$.

\FIXME{finish the description and the algorithm}

\begin{algorithm}
\caption{Procedure \peupdate}
\label{alg:pe.update}

\begin{algorithmic}
\REQUIRE $H_e$, the last history appended to $\unf{N}$
\ENSURE \pe{} is updated with all new histories for events $e'$, with $e
\nearrow e'$

\STATE \peupdatecont ($H_e$)

\STATE \peupdatepostexisting ($H_e$)
\STATE \peupdatepostnew ($H_e$, $m$)

\end{algorithmic}
\end{algorithm}


\begin{algorithm}
\caption{Procedure \peupdatecont}
\label{alg:pe.update.cont}

\begin{algorithmic}
\REQUIRE A new history $H_e$
\ENSURE \pe{} is updated with all new histories for events $e'$, with $\cont{e}
\cap \pre{e'} \neq \emptyset$ and $\post{e} \cap \pre{e'} = \emptyset$

\FORALL {$e'$ such that $\cont{e} \cap \pre{e'} \neq \emptyset$ and $\post{e}
\cap \pre{e'} = \emptyset$}
\FORALL {$H_{e'} \in \hist{e'}$}

\STATE $f \leftarrow $ \TRUE
\FORALL {$H_{e''}$ such that $(H_{e'}, H_{e''})$ is an edge of $\h$ and $e \neq
e''$}
\STATE $f \leftarrow f \land \lnot H_e \# H_{e''}$
\ENDFOR

\STATE $H \leftarrow H_e \cup H_{e'}$
\IF{$f$ and not $\text{\pehistdup{}} (H)$}
\STATE Append to $\h$ a new node $H$, labeled by $e'$
\STATE Append to $\h$ edges $(H, H_{e''})$ for any $H_{e''}$ such that there
exists some edge $(H_{e'}, H_{e''})$ already present in $\h$
\STATE Append to $\h$ the edge $(H, H_e)$
\STATE Append to \pe{} the new history $H$
\ENDIF

\ENDFOR
\ENDFOR

\end{algorithmic}
\end{algorithm}

\begin{algorithm}
\caption{Procedure \peupdatepostexisting}
\label{alg:pe.update.post.existing}

\begin{algorithmic}
\REQUIRE A new history $H_e$
\ENSURE \pe{} is updated with all new histories for events $e'$, with $\post{e}
\cap (\pre{e'} \cup \cont{e'}) \neq \emptyset$ and $e'$ is already present in
the unfolding.

\FORALL {$e'$ such that $\post{e} \cap (\pre{e'} \cup \cont{e'}) \neq
\emptyset$}

\STATE $\{c_1, \ldots, c_n\} \leftarrow \pre{e'} \cup \cont{e'}$, assuming that
$e = \pre{c_1}$
\STATE $\{e_1, \ldots, e_m\} \leftarrow \pre{\{c_1, \ldots, c_n\}}$, assuming
that $e = e_1$.

\FORALL {histories $H_{e_1}, \ldots, H_{e_m}$ for events $e_1, \ldots, e_m$
such that $H_{e_1} = H_e$}
\STATE $f \leftarrow \TRUE$
\FOR {$i = 1$ to $m$}
\STATE $f \leftarrow f \land H_{e_i} \text{ doesn't consume any condition from
} c_1, \ldots, c_n$
\FOR {$j = 1$ to $i - 1$}
\STATE $f \leftarrow f \land \lnot H_{e_j} \# H_{e_i}$
\ENDFOR
\ENDFOR

\STATE $H = \{e'\} \cup H_{e_1} \cup \ldots \cup H_{e_m}$
\IF {$f$ and not $\text{\pehistdup{}} (H)$}
\STATE Append to $\h$ a new node $H$, labeled by $e'$
\STATE Append to $\h$ new edges $(H, H_{e_i})$ for $1 \le i \le m$
\STATE Append to \pe{} the history $H$
\ENDIF
\ENDFOR
\ENDFOR

\end{algorithmic}
\end{algorithm}

\begin{algorithm}
\caption{Procedure \peupdatepostnew}
\label{alg:pe.update.post.new}

\begin{algorithmic}
\REQUIRE A new history $H_e$
\ENSURE \pe{} is updated with all new events $e'$ such that $t = \origin{e'}$
and $(\pre{t} \cup \cont{t}) \cap \origin{\post{e}} \neq \emptyset$ and $e'$
was still not present in the unfolding.

\STATE \FIXME{}

\end{algorithmic}
\end{algorithm}


\begin{algorithm}
\caption{Procedure \pehistdup}
\label{alg:pe.hist.dup}

\begin{algorithmic}
\REQUIRE A new history $H$ and the event $e$ to which $H$ is associated
\ENSURE Returns \TRUE{} if the history is already present in the current state
of the unfolding or \pe{}; otherwise returns \FALSE{}

\IF {$H$ is associated to $e$ and present in the unfolding or in \pe{}}
\RETURN \TRUE
\ELSE
\RETURN \FALSE
\ENDIF

\end{algorithmic}
\end{algorithm}

\begin{algorithm}
\caption{Procedure \peeventdup}
\label{alg:pe.event.dup}

\begin{algorithmic}
\REQUIRE A presumably new event $e$
\ENSURE Returns \TRUE{} the event is a duplicate in the current state of the
unfolding; \FALSE{} otherwise

\IF {there exists $e'$ in the unfolding or \pe{} such that $\origin{e'} =
\origin{e}$ and $\pre{e'} = \pre{e}$ and $\cont{e'} = \cont{e}$}
\RETURN \TRUE
\ELSE
\RETURN \FALSE
\ENDIF

\end{algorithmic}
\end{algorithm}

\newpage

\begin{definition}
Given two configurations $C_1, C_2 \in \conf{\unf{N}}$, we say that $C_1$ can
\emph{evolve} to $C_2$, written $C_1 \evolves C_2$, iff $C_1 \subseteq C_2$ and
for all $e_1 \in C_1$ and all $e_2 \in C_2 \setminus C_1$ we have $\lnot (e_2
\nearrow e_1)$.
\end{definition}

\begin{definition}
Let $C_1, C_2 \in \conf{\unf{N}}$ be two configurations of the full unfolding.
We say that both configurations are \emph{in conflict}, written $C_1 \# C_2$,
iff $\lnot (C_1 \evolves C_1 \cup C_2)$ or $\lnot (C_2 \evolves C_1 \cup C_2)$.
\end{definition}

Another characterization of the conflict relation between configurations is to
say that $C_1 \# \, C_2$ if and only if there exists $e_1 \in C_1$ and $e_2 \in
C_2 \setminus C_1$ with $e_2 \nearrow e_1$ or the symmetric condition holds.
We will make extensive use of this equivalence in the sequel.

\begin{definition}
Let $C$ be a configuration of the full unfolding $\unf{N} = (S', T', F', C',
m')$.  We define the \emph{cut} of $C$ as $$\cut{C} \: = \: m \: \cup \:
\bigcup_{e \in C} \post{e} \: \setminus \: \bigcup_{e \in C} \pre{e}$$.
\end{definition}

\begin{definition}
Let $\unf{N} = (S', T', F', C', m')$ be the full unfolding of some net $N$.  We
say that a pair $(H, c)$ is an \emph{enriched condition} if either for some
event $e \in T'$, we have $H \in \hist{e}$ and $c \in \post{e}$, or $H =
\emptyset$ and $c \in m'$.  Additionally, we write $\conds{\unf{N}}$ to denote
the set of enriched conditions of $\unf{N}$, that is, the set of enriched
conditions $(H_e, c)$ for any $e \in T'$ plus all enriched conditions
$(\emptyset, c$ for $c \in m'$.
\end{definition}

Enriched events are defined similarly:

\begin{definition}
Let $\unf{N} = (S', T', F', C', m')$ be the full unfolding of some net $N$.  We
say that a pair $(H, e)$ is an \emph{enriched event} if for some event $e \in
T'$, we have $H \in \hist{e}$.  Additionally, we write $\events{\unf{N}}$ to
denote the set of enriched events of $\unf{N}$, that is, the set of enriched
events $(H_e, e)$ for any $e \in T'$.
\end{definition}

We will let $\varepsilon$ to range over enriched events and $\rho, \sigma,
\ldots$ to range over enriched conditions.  If $\varepsilon = (H, e)$, we will
write $\varepsilon^H$ to denote $H$ and $\varepsilon^e$ to denote $e$.  In a
similar way, if $\rho = (H, c)$, we will write $\rho^H$ to denote $H$ and
$\rho^c$ to denote $c$.

\FIXME{Define (and comment) $H \sem c$ as $\emptyset$ if $\pre c = \emptyset$
or as $H \sem e$ if $\{e\} = \pre c$.}  We extend now the notation for preset,
postset and context of events and conditions to enriched events and enriched
conditions.

\begin{definition}
Let $\unf N = (S', T', F', C', m')$ be the full unfolding of some net $N$.  We
define the \emph{preset}, \emph{postset} and \emph{context} of an enriched
event $(H_e, e) \in \events{\unf N}$ and an enriched condition $(H_e, c) \in
\conds{\unf N}$:

\begin{align*}
\post{(H, e)} & \; \bydef = \; \{ (H, c) \in \conds{\unf N} \mid c \in \post e
\} \\
\pre{(H, e)} & \; \bydef = \; \{ (H', c) \in \conds{\unf N} \mid c \in \pre e
\land H' = H \sem c \} \\
\cont{(H, e)} & \; \bydef = \; \{ (H', c) \in \conds{\unf N} \mid c \in \cont e
\land H' = H \sem c \} \\
\post{(H, c)} & \; \bydef = \; \{ (H', e) \in \events{\unf N} \mid (H, c) \in
\pre{(H', e)} \} \\
\pre{(H, c)} & \; \bydef = \; \{ (H, e) \in \events{\unf N} \mid e \in \pre c
\} \\
\cont{(H, c)} & \; \bydef = \; \{ (H', e) \in \events{\unf N} \mid (H, c) \in
\cont{(H', e)} \} \\
\end{align*}
\end{definition}

We are now ready to define the concurrency relation between enriched
conditions.

\begin{definition}
The \emph{concurrency relation} $\mathord\parallel \subseteq \conds{\unf N}
\times \conds{\unf N}$ between extended conditions is defined as: $$ (H, c)
\parallel (H', c') \iff \lnot (H \confl H') \land \{c, c'\} \subseteq \cut{H
\cup H'}$$
\end{definition}

For the sake of simplicity in the following developments, we characterize the
concurrency relation in terms of \FIXME{...}

\begin{remark}
\label{rmk:the.statement}
The statement $(H, c) \parallel (H', c')$ is equivalent to the conjunction of
the next four statements:
\begin{enumerate}
\item $\lnot (\exists e_1 \in H,\, \exists e_2 \in H' \setminus H,\, e_2
\nearrow e_1)$
\item $\lnot (\exists e_1 \in H',\, \exists e_2 \in H \setminus H',\, e_2
\nearrow e_1)$
\item $\lnot (\exists e_1 \in H,\, c' \in \pre{e_1})$
\item $\lnot (\exists e_1 \in H',\, c \in \pre{e_1})$
\end{enumerate}
\end{remark}

In other words, \rrmk{the.statement} establishes that $(H, c) \parallel (H',
c')$ if and only if $H$ can evolve to $H \cup H'$ (1), $H'$ can evolve to $H
\cup H'$ (2) and none of any histories consumes the condition generated by the
other history (3 and 4).  Note that the conjunction of conditions 1 and 2 is
equivalent to $\lnot (H \confl H')$ while the conjunction of conditions 3 and 4
is equivalent to $c, c' \in \cut{H \cup H'}$.

\begin{lemma}
\label{lem:let.H}
Let $H, H' \in \hist{e}$ be two histories of some event $e$.  If $H \not= H'$,
then $H \confl H'$.
\end{lemma}

\begin{proof}
\FIXME{Are we using this lemma? Copy from the notebook!}
\end{proof}

\begin{lemma}
\label{lem:two.histories}
Let $H, H'$ be two histories with $H' \prec H$, $e = \event{H}$ and $e \in H'$.
Then $H \confl H'$.
\end{lemma}

\begin{proof}
We claim that $H \not\sqsubseteq H'$.  Indeed, if it were the case that $H
\sqsubseteq H'$, and provided that $\prec$ is antisymmetric, we would have $H
\prec H'$, a contradiction to the hypothesis $H' \prec H$.  So $H
\not\sqsubseteq H'$ holds.  Making use of \FIXME{def $\sqsubseteq$}, one can
write the next equivalence: $$H \not\sqsubseteq H' \iff H \not\subseteq H' \lor
\exists e_1 \in H,\, \exists e_2 \in H' \setminus H,\, e_2 \nearrow e_1$$  As
$H \not\sqsubseteq H'$ holds, the right-hand side also holds.  We show that
both sides of the disjunction in the right-hand size implies $H \confl H'$.

Clearly, $\exists e_1 \in H,\, \exists e_2 \in H' \setminus H,\, e_2 \nearrow
e_1$ implies $H \confl H'$.  On the other hand, $H \not\subseteq H'$ implies
that we can find some event $e'' \in H \setminus H'$.  Now notice that, as $e''
\in H$, it holds that $e'' \nearrow^*_H e$.  Event $e$ and $e''$ must be
different because $e \in H'$ by hypothesis and $e'' \notin H'$.  Consequently
it also holds that $e'' \nearrow^+_H e$.  This means that we can find a finite
number of events $e_1, \ldots, e_n \in H$ with $e'' = e_1 \nearrow e_2 \nearrow
\ldots \nearrow e_n = e$.  It is easy to see that there must exist some pair of
events $e_l, e_{l+1}$ for some $1 \le l < k$ with $e_l \in H \setminus H'$ and
$e_{l+1} \in H'$.  By construction, $e_l \nearrow e_{l+1}$ and hence $H \confl
H'$.
\end{proof}


\FIXME{Give motivation to state this theorem}

\begin{theorem}
Let $\rho = (H, c)$, $\rho' = (H', c')$ be two different enriched conditions
such that $\event{H} = e$, $\event{H'} = e'$ and $H' \prec H$ for an adequate
order $\prec$.  Then, we have that $$\rho \parallel \rho' \iff
\bigwedge_{\rho_i \in \pre{\pre\rho}} \rho_i \parallel \rho' \; \land \;
\bigwedge_{\sigma_j \in \cont{\pre\rho}} \sigma_j \parallel \rho' \; \land \;
(\rho' \notin \pre{\pre \rho}) \; \land \; \lnot \exists e'' \in H' \setminus
H,\, \cont{e''} \cap \pre e \not= \emptyset$$
\end{theorem}

\begin{proof}
Let $\pre{\pre\rho} = \{\rho_1, \ldots, \rho_n\}$ and $\cont{\pre\rho} =
\{\sigma_1, \ldots, \sigma_m\}$.  We let variable $\rho_i = (H_i, c_i)$ to
range the set $\pre{\pre \rho}$ and variable $\sigma_j = (H_j, c_j)$ to range
the set $\cont{\pre \rho}$, with $1 \le i \le n$ and $1 \le j \le m$.  For any
$\rho_i$ and $\sigma_j$, we let $e_i = \event{H_i}$ and $e_j = \event{H_j}$.
Note also that, by definition, $H_i = H\sem{e_i} = H\sem{c_i}$ and that $H_j =
H\sem{e_j} = H\sem{c_j}$.  It is easy also to see that for any pair $\rho_1,
\rho_2 \in \pre{\pre\rho} \cup \cont{\pre\rho}$ we have $\rho_1 \parallel
\rho_2$.  Finally note that, also by construction, $H = \{e\} \cup
\bigcup_{\rho_i \in \pre{\pre\rho}} \rho^H_i \cup \bigcup_{\sigma_j \in
\cont{\pre\rho}} \sigma^H_j$.

We prove in the sequel both directions of the co-implication. \FIXME{Give
outline of the proof}

\begin{enumerate}[(a)]
\item $\rho \parallel \rho' \implies \bigwedge_{\rho_i \in \pre{\pre
\rho}} \rho_i \parallel \rho'$  It is enough to prove that $\rho \parallel
\rho' \implies \rho_i \parallel \rho'$ for \emph{some} $\rho_i$.  As we do not
make any assumption on $\rho_i$, the proof will be valid for \emph{all}
$\rho_i$.

Assume that $\rho \parallel \rho'$ and that, for a proof by contradiction,
$\lnot (\rho_i \parallel \rho')$ for some $\rho_i \in \pre{\pre\rho}$.  Then we
know that at least one of the four statements in \rrmk{the.statement} must be
false when regarding $\rho_i$ and $\rho'$.  We proceed by cases:

\begin{enumerate}[1.]
\item Assume that there exist events $e_1 \in H_i$ and $e_2 \in H' \setminus
H_i$ with $e_2 \nearrow e_1$.  Two cases are possible: either $e_2 \in H$ or
$e_2 \notin H$.  If $e_2 \notin H$, as $H_i \subseteq H$, we have that $e_1 \in
H$, $e_2 \in H' \setminus H$ and $e_2 \nearrow e_1$, which implies $H \confl
H'$.  This is a contradiction to $\rho \parallel \rho'$.  So, let us assume
that $e_2 \in H$.  As $H = \{e\} \cup \bigcup_{\rho_i \in \pre{\pre\rho}}
\rho^H_i \cup \bigcup_{\rho_j \in \cont{\pre \rho}} \rho^H_j$, several
sub-cases are possible:

\begin{itemize}
\item $e_2 \in \rho^H_{i'}$ for some $\rho_{i'} \in \pre{\pre\rho}$ with $i'
\not= i$.  Again, we have $e_1 \in H_i$, $e_2 \in \rho^H_{i'} \setminus H_i$
and $e_2 \nearrow e_1$.  This implies $\rho^H_{i'} \confl H_i$, which is a
contradiction to $\rho_i \parallel \rho_{i'}$.

\item $e_2 \in \sigma^H_j$ for some $\sigma_j \in \cont{\pre\rho}$.  Likewise,
we have $e_1 \in H_i$, $e_2 \in \sigma^H_j \setminus H_i$, $e_2 \nearrow e_1$
and $\sigma^H_j \confl H_i$, which is a contradiction to $\rho_i \parallel
\sigma_j$.

\item $e_2 = e$.  By definition \FIXME{history} we have $e_1 \nearrow^*_{H_i}
e_i$ (recall that $e_i = \event{H_i}$).  By construction of $H$, we have $e_i
\nearrow e$.  By hypothesis, we have $e_2 = e$ and also by hypothesis we have
$e_2 \nearrow e_1$.  This leads to the cycle $e_1 \nearrow^*_{H_i} e_i \nearrow
e = e_2 \nearrow e_1$ in the asymmetric conflict relation.  Now notice that
$e_1, e_i, e \in H$, and that $H_i \subseteq H$.  Therefore, we have a loop in
the asymmetric conflict relation restricted to the history $H$, which is a
contradiction to the fact that $H$ is a history.
\end{itemize}

\item Assume that there exist $e_1 \in H'$ and $e_2 \in H_i \setminus H'$ with
$e_2 \nearrow e_1$.  Then, as $H_i \subseteq H$, we have that $e_2 \in H
\setminus H'$.  This together with $e_1 \in H'$ and $e_2 \nearrow e_1$ implies
$H \confl H'$, which is a contradiction to $\rho \parallel \rho'$.

\item Assume that there exists $e_1 \in H_i$ such that $c' \in \pre{e_1}$.
Intuitively, this means that $H_i$ consumes $c'$.  As $H_i \subseteq H$, also
$H$ consumes $c'$, which leads to a contradiction of $\rho \parallel \rho'$.

\item Assume that there exists $e_1 \in H'$ such that $c_i \in \pre{e_1}$.  We
have two cases, either $e_1 = e$ or $e_1 \not= e$.  If we assume that $e_1
\not= e$, then we have that $\pre e \cap \pre{e_1} \not= \emptyset$.  This
implies that $e \nearrow e_1$ and that $e_1 \nearrow e$.  In turn, it implies
that $e_1 \notin H$ (otherwise we would have a loop in the relation $\nearrow$
restricted to $H$).  So assuming that $e_1 \not= e$ would lead to conclude that
$e_1 \in H' \setminus H$ and that $e_1 \nearrow e$, with $e \in H$, which
implies that $H \confl H'$, a contradiction to $\rho \parallel \rho'$.

Therefore, let us assume that $e_1 = e$ and, consequently, that $e \in H'$.
As, by hypothesis $H' \prec H$, we can apply \rlem{two.histories} and conclude
that $H \confl H'$, a contradiction to $\rho \parallel \rho'$.
\end{enumerate}

\item $\rho \parallel \rho' \implies \bigwedge_{\sigma_j \in \cont{\pre\rho}}
\sigma_j \parallel \rho'$  As in (a), it is still enough to prove that the
statement holds for just one $\sigma_j$.  In particular, we will prove that
$\rho \parallel \rho' \implies \sigma_j \parallel \rho'$ for some $\rho_j \in
\cont{\pre\rho}$.  As we make no assumption about $\sigma_j$, the argument will be
valid for \emph{all} $\sigma_j$.

We reason by contradiction.  Assume that both $\rho \parallel \rho'$ and $\lnot
(\sigma_j \parallel \rho')$ hold, for some $\sigma_j \in \cont{\pre\rho}$.  As
$\lnot (\sigma_j \parallel \rho')$, we know that at least one of the four
statements in \rrmk{the.statement} must not hold.  In the following, we see
that we can find a contradiction in each case.

\begin{enumerate}[1.]
\item Assume that there exist events $e_1 \in H_j$ and $e_2 \in H' \setminus
H_j$ with $e_2 \nearrow e_1$.  As in (a.1), two cases are possible: either $e_2
\in H$ or $e_2 \notin H$.  If $e_2 \notin H$, as $H_j \subseteq H$, we have
that $e_1 \in H$, $e_2 \in H' \setminus H$ and $e_2 \nearrow e_1$, which
implies $H \confl H'$.  This is a contradiction to $\rho \parallel \rho'$.  So,
let us assume that $e_2 \in H$.  As $H = \{e\} \cup \bigcup_{\rho_i \in
\pre{\pre\rho}} \rho^H_i \cup \bigcup_{\sigma_j \in \cont{\pre \rho}}
\sigma^H_j$, several sub-cases are possible:

\begin{enumerate}
\item $e_2 \in \sigma^H_{j'}$ for some $\sigma_{j'} \in \cont{\pre\rho}$ with
$j' \not= j$.  Again, we have $e_1 \in H_j$, $e_2 \in \sigma^H_{j'} \setminus
H_j$ and $e_2 \nearrow e_1$.  This implies $\sigma^H_{j'} \confl H_j$, which is
a contradiction to $\sigma_j \parallel \sigma_{j'}$.

\item $e_2 \in \rho^H_i$ for some $\rho_i \in \pre{\pre\rho}$.  Likewise, we
have $e_1 \in H_j$, $e_2 \in \rho^H_i \setminus H_j$, $e_2 \nearrow e_1$ and
$\rho^H_i \confl H_j$, which is a contradiction to $\sigma_j \parallel \rho_i$.

\item $e_2 = e$.  Recall that $e_j = \event{H_j}$.  By definition
\FIXME{history} we have $e_1 \nearrow^*_{H_j} e_j$.  By construction of $H$, we
have $e_j \nearrow e$.  By hypothesis, we have $e_2 = e$ and also by hypothesis
we have $e_2 \nearrow e_1$.  This leads to the cycle $e_1 \nearrow^*_{H_j} e_j
\nearrow e = e_2 \nearrow e_1$ in the asymmetric conflict relation.  Now notice
that $e_1, e_j, e \in H$, and that $H_j \subseteq H$.  Therefore, we have a
loop in the asymmetric conflict relation restricted to the history $H$, which
is a contradiction to the fact that $H$ is a history.
\end{enumerate}

\item Assume that there exist $e_1 \in H'$ and $e_2 \in H_j \setminus H'$ with
$e_2 \nearrow e_1$.  The same argument as in (a.2) is still applicable here,
changing $H_i$ by $H_j$.

\item Assume that there exists $e_1 \in H_j$ such that $c' \in \pre{e_1}$.
Same argument as in (a.3), substituting $H_i$ by $H_j$.

\item Assume that there exists $e_1 \in H'$ such that $c_j \in \pre{e_1}$.  We
then know that $\cont{e} \cap \pre{e_1} \not= \emptyset$.  \FIXME{As we deal
with nets in which for any event $e''$ it is not the case that $\cont{e''} \cap
\pre{e''} \not= \emptyset$, we have to assume that $e \not= e_1$.} Notice also
that $e \nearrow e_1$.  We claim now that $e \in H'$.  By contradiction, if it
were the case that $e \notin H'$, we would have that $e \in H \setminus H'$,
$e_1 \in H'$ and $e \nearrow e_1$, and consequently $H \confl H'$.

Therefore we know that $e \in H'$.  By hypothesis we also know that $H' \prec
H$.  Under this assumptions we can apply \rlem{two.histories} and conclude that
$H \confl H'$, a contradiction to $\rho \parallel H'$.
\end{enumerate}

\item $\rho \parallel \rho' \implies \rho' \notin \pre{\pre\rho}$  It is easy
to see that it cannot be the case that $\rho' \in \pre{\pre\rho}$ holds if we
assume $\rho \parallel \rho'$.  Indeed, if $\rho' \in \pre{\pre\rho}$ we have
that $\rho' = \rho_i = (H_i, c_i)$ for some $\rho_i \in \pre{\pre\rho}$.  But
notice that event $c_i \in \pre e$, that is, event $e$ consumes $c_i$.  This
implies that the third statement of \rrmk{the.statement} doesn't hold with
regard to $\rho$ and $\rho'$ and we have $\lnot (\rho \parallel \rho')$, a
contradiction.

\item $\rho \parallel \rho' \implies \lnot \exists e'' \in H' \setminus
H,\, \cont{e''} \cap \pre e \not= \emptyset$  Assume, for an argument by
contradiction, that it is the case that there exists $e'' \in H' \setminus H$
such that $\cont{e''} \cap \pre e \not= \emptyset$.  In consequence, we have
$e'' \nearrow e$, with $e \in H$ and $e'' \in H' \setminus H$, which implies $H
\confl H'$, a contradiction to $\rho \parallel \rho'$.

\item $\rho \parallel \rho' \Longleftarrow \bigwedge_{\rho_i \in \pre{\pre
\rho}} \rho_i \parallel \rho' \land \bigwedge_{\rho_j \in \cont{\pre \rho}}
\rho_j \parallel \rho' \land (\rho' \notin \pre{\pre \rho}) \land \lnot \exists
e'' \in H' \setminus H,\, \cont{e''} \cap \pre e \not= \emptyset$  We prove now
the opposite direction of the theorem.  We assume the right-hand side of the
implication and the negation of the left-hand side.  As $\lnot (\rho \parallel
\rho')$, one of the statements of \rrmk{the.statement} must be false:

\begin{enumerate}[1.]
\item Assume that there exist events $e_1 \in H$ and $e_2 \in H' \setminus H$
with $e_2 \nearrow e_1$.  Recall that $H = \{e\} \cup \bigcup_{\rho_i \in
\pre{\pre\rho}} \rho^H_i \cup \bigcup_{\sigma_j \in \cont{\pre\rho}}
\sigma^H_j$.  We regard $e_1$ and reason by cases:

\begin{itemize}
\item Assume that $e_1 \in H_i$ for some $\rho_i \in \pre{\pre\rho}$. As $H_i
\subseteq H$, we still have $e_2 \in H' \setminus H_i$, and hence $H_i \confl
H'$, a contradiction to $\rho_i \parallel \rho'$.

\item Assume that $e_1 \in H_j$ for some $\sigma_j \in \cont{\pre\rho}$.  In
the same, way we can see that $H_j \confl H'$, a contradiction.

\item Finally, assume that $e_1 = e$ and, consequently, $e_2 \nearrow e$.
\FIXME{Definition $\nearrow$} provides three cases:

\begin{itemize}
\item Assume that $\pre{e_2} \cap \pre e \not= \emptyset$.  Under this
assumption, $e_2 \in H'$ clearly consumes $c_i$ for one $\rho_i \in
\pre{\pre\rho}$, which implies that $\lnot (\rho' \parallel \rho_i)$, a
contradiction.

\item Assume that $e_2 < e$.  We know that $H$ is a history, and by definition
contains all events $e'' < e$.  As $e_2$ is one such event, we have $e_2 \in
H$, which is a contradiction to the assumption $e_2 \in H' \setminus H$.

\item Assume that $\cont{e_2} \cap \pre e \not= \emptyset$.  This is a
contradiction to the last conjunction in hypothesis of the statement that we
are proving.
\end{itemize}
\end{itemize}

\item Assume that there exist events $e_1 \in H'$ and $e_2 \in H \setminus H'$
with $e_2 \nearrow e_1$.  Using the same arguments as in (e.1), we can
immediately discard the cases where $e_2 \in H_i$ for some $\rho_i \in
\pre{\pre\rho}$ or $e_2 \in H_j$ for some $\sigma_j \in \cont{\pre\rho}$.  We
assume, hence, that $e_2 = e$.  Definition of \FIXME{$\nearrow$} gives us three
cases to examine in the relation $e \nearrow e_1$:

\begin{itemize}
\item Assume that $\pre{e} \cap \pre{e_1} \not= \emptyset$.  Then event $e_1
\in H'$ consumes $c_i$ for some $\rho_i \in \pre{\pre\rho}$, which leads to the
contradiction $\lnot (\rho_i \parallel \rho)$.

\item Assume that $e < e_1$.  As $e_1 \in H'$ and $H'$ is a history, we should
have $e \in H'$, while by hypothesis $e = e_2 \in H \setminus H'$ and hence $e
\notin H'$.

\item Assume that $\cont e \cap \pre{e_1} \not= \emptyset$.  Then $c_j \in
\pre{e_1}$ for some $\sigma_j \in \cont{\pre\rho}$.  As $e_1 \in H'$, we have
that $\lnot (\sigma_j \parallel \rho')$.  This is a contradiction.
\end{itemize}

\item Assume that there exists $e_1 \in H$ such that $c' \in \pre{e_1}$.  If we
assume $e_1 \in H_i$ for some $\rho_i \in \pre{\pre\rho}$, we will find the
contradiction $\lnot (\rho_i \parallel \rho')$.  Similarly, if we assume $e_1
\in H_j$ for some $\sigma_j \in \cont{\pre\rho}$ we will reach the
contradiction $\lnot (\sigma_j \parallel \rho')$.  So the only case that we
examine is when $e_1 = e$.  If $c' \in \pre e$, we have that $c' = c_i$ for
some $\rho_i \in \pre{\pre\rho}$.  As $|\pre{c''}| = 1$ for any $c''$, we have
that $\event{H_i} = \event{H'}$.  Intuitively, this means that $H_i$ and $H'$
are histories for the same event.  Furthermore, they are different by
hypothesis ($\rho' \notin \pre{\pre\rho}$ is one of the hypothesis).  Under
this assumptions, we can apply \rlem{let.H} and conclude that $H' \confl H_i$,
which is a contradiction.

\item Assume that there exists $e_1 \in H'$ such that $c \in \pre{e_1}$.  Then
$e \in H'$, and $H'$ consumes any $c_i$ for $\rho_i \in \pre{\pre\rho}$, which
is a contradiction.
\end{enumerate}
\end{enumerate}
\end{proof}

\end{document}

 % vim:syn=tex:spell:

