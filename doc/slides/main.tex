
% vi:fenc=utf8:spell:spl=en:

\documentclass[table,red,11pt]{beamer}

\usepackage[utf8]{inputenc}
\usepackage[english]{babel}
\usepackage{graphicx}
\usepackage{algorithmic}
\usepackage{figlatex}
\usepackage{url}
\usepackage{mydefs}
%\usepackage{pstricks}

% appearance (i)
%\usetheme{Warsaw}
%\usecolortheme{seahorse} %o beaver, lily
%\usefonttheme{structuresmallcapsserif}
\useoutertheme{infolines}

% appearance (ii)
%\usetheme{Frankfurt}
%\usecolortheme{lily} %o beaver, lily
%\useoutertheme{infolines} %<- change this

% misc
%\linespread{1.3}
\newcommand{\confl}{\ensuremath{\mathord{\#}}}
\newcommand{\peupdate}{\mbox{\sc Pe-Update}}
\newcommand{\iscutoff}{\mbox{\sc Is-Cutoff}}
\newcommand{\hist}[1]{\ensuremath{\mathsf{Hist}(#1)}}


% title, author and dates
\title[]{Implementation of a complete prefix unfolder for contextual nets}
\author[César Rodríguez]{César Rodríguez\\{\footnotesize
\url{cesar.rodriguez@lsv.ens-cachan.fr}}}
\institute[LSV]{LSV, ENS Cachan}
\date{September 2010}

\begin{document}

% *****************************************************************************
\begin{frame}
\thispagestyle{empty}
\titlepage
\end{frame}

%% ****************************************************************************
%\begin{frame}{Graph Unfolding}
%
%\uncover<1->{\centerline{\includegraphics{fig/graph-unfolding.fig}}}
%
%\begin{itemize}
%\uncover<1->{\item Each node of $U$ represents \alert{a path} in $G$}
%\uncover<1->{\item $U$ is the \alert{full unfolding} of $G$}
%\uncover<1>{\item $U$ is usually infinite}
%\end{itemize}
%\end{frame}

% *****************************************************************************
\begin{frame}[t]{Contextual Nets}
\vspace{1cm}
\begin{columns}[T]

\column{.3\textwidth}
\centerline{\includegraphics{fig/contextual-net.fig}}

\column{.7\textwidth}
\begin{itemize}
\item $N = \langle P, T, F, C, m_0 \rangle$
\only<1>{\begin{itemize}
\item $P$: finite set of \emph{places}
\item $T$: finite set of \emph{transitions}
\item $F \subseteq P \times T \cup T \times P$: \emph{flow relation}
\item $C \subseteq P \times T$: \alert{context relation} (read arcs)
\item $m_0 \subseteq P$: \emph{initial marking}
\vspace{4ex}
%\item Remark: we consider only \alert{1-safe} nets
\item Context of $t$: $\cont t = \{p \in P \mid (p, t) \in C\}$
\end{itemize}
}

\item<2-> Enabled transition
\begin{itemize}
\item Transition $t \in T$ is \alert{enabled} at marking $m \subseteq P$ iff
$\pre t \cup \cont t \subseteq m$
\item E.g.: $t_2$ enabled at $\{p_3, p_1\}$
\item Remark: $p_3$ \alert{remains marked} when $t_2$ fires
\end{itemize}

\item<3-> Run of $N$
\begin{itemize}
\item A \alert{run} of $N$ is a sequence $t_1, \ldots, t_n$ of transitions such
that $t_{i+1}$ is enabled at the marking generated by $t_i$
\item E.g.: $t_1, t_2, t_3$
\end{itemize}
\end{itemize}
\end{columns}
\end{frame}

% ****************************************************************************
\begin{frame}[t]{Contextual Net Unfoldings}

\only<1>{\centerline{\includegraphics{fig/contextual-unfoldings-1.fig}}}
\only<1>{
\begin{itemize}
\item Unfolding $\unf N$ acyclic, possibly infinite
\item Places/transitions in $\unf N$ are called
\alert{conditions}/\alert{events}
\end{itemize}
}

%\only<2>{\centerline{\includegraphics{fig/contextual-unfoldings-2.fig}}}
\only<2>{\centerline{\includegraphics{fig/contextual-unfoldings-3.fig}}}
\only<3>{\centerline{\includegraphics{fig/contextual-unfoldings-5.fig}}}
\only<3>{\centerline{Copy of the initial marking}}
\only<4>{\centerline{\includegraphics{fig/contextual-unfoldings-4.fig}}}
\only<5>{\centerline{\includegraphics{fig/contextual-unfoldings-6.fig}}}
%\only<7>{\centerline{\includegraphics{fig/contextual-unfoldings-7.fig}}}
%\only<8>{\centerline{\includegraphics{fig/contextual-unfoldings-8.fig}}}
\only<6>{\centerline{\includegraphics{fig/contextual-unfoldings-9.fig}}}
%\only<10>{\centerline{\includegraphics{fig/contextual-unfoldings-10.fig}}}
\only<7>{\centerline{\includegraphics{fig/contextual-unfoldings-11.fig}}}
\only<8>{\centerline{\includegraphics{fig/contextual-unfoldings-12.fig}}}
\only<9>{\centerline{\includegraphics{fig/contextual-unfoldings-13.fig}}}

\only<10>{\centerline{\includegraphics{fig/contextual-unfoldings-14.fig}}}
\only<11>{\centerline{\includegraphics{fig/contextual-unfoldings-15.fig}}}
%\only<16>{\centerline{\includegraphics{fig/contextual-unfoldings-16.fig}}}
%\only<17>{\centerline{\includegraphics{fig/contextual-unfoldings-17.fig}}}
\only<12>{\centerline{\includegraphics{fig/contextual-unfoldings-18.fig}}}
%\only<14>{\centerline{\includegraphics{fig/contextual-unfoldings-19.fig}}}
\only<13>{\centerline{\includegraphics{fig/contextual-unfoldings-20.fig}}}
\only<13>{\centerline{Fresh copy of $p_2$!}}
\only<14>{\centerline{\includegraphics{fig/contextual-unfoldings-21.fig}}}
\only<15>{\centerline{\includegraphics{fig/contextual-unfoldings-22.fig}}}
\only<16>{\centerline{\includegraphics{fig/contextual-unfoldings-23.fig}}}
\only<17>{\centerline{\includegraphics{fig/contextual-unfoldings-24.fig}}}
%\only<25>{\centerline{\includegraphics{fig/contextual-unfoldings-25.fig}}}
%\only<26>{\centerline{\includegraphics{fig/contextual-unfoldings-26.fig}}}
\only<18>{\centerline{\includegraphics{fig/contextual-unfoldings-27.fig}}}
%\only<28>{\centerline{\includegraphics{fig/contextual-unfoldings-28.fig}}}
%\only<20>{\centerline{\includegraphics{fig/contextual-unfoldings-29.fig}}}
%\only<21>{\centerline{\includegraphics{fig/contextual-unfoldings-30.fig}}}
%\only<22>{\centerline{\includegraphics{fig/contextual-unfoldings-31.fig}}}
%\only<23>{\centerline{\includegraphics{fig/contextual-unfoldings-32.fig}}}
%\only<23>{\centerline{A copy of $t_3$ (namely, $e_3$) is already present}}
%\only<33>{\centerline{\includegraphics{fig/contextual-unfoldings-33.fig}}}
%\only<34>{\centerline{\includegraphics{fig/contextual-unfoldings-34.fig}}}
\only<19>{\centerline{\includegraphics{fig/contextual-unfoldings-35.fig}}}
%\only<36>{\centerline{\includegraphics{fig/contextual-unfoldings-36.fig}}}

%\only<20>{\centerline{\includegraphics{fig/contextual-unfoldings-37.fig}}}
%\only<38>{\centerline{\includegraphics{fig/contextual-unfoldings-38.fig}}}
%\only<39>{\centerline{\includegraphics{fig/contextual-unfoldings-39.fig}}}
%\only<40>{\centerline{\includegraphics{fig/contextual-unfoldings-40.fig}}}
\uncover<20->{\centerline{\includegraphics{fig/contextual-unfoldings-41.fig}}}


\begin{itemize}
\item<20-> $\unf N$: \alert{full unfolding}, limit of this construction,
usually infinite.
\item<20-> $\pref N$: unfolding prefix, any intermediate state, finite.
\end{itemize}
\end{frame}

% ****************************************************************************
\begin{frame}{Contextual Net Unfoldings Exploit Concurrency}

\begin{itemize}
\item \alert{Complete prefix}: represents all reachable markings of $N$
\item Petri Net are a state space \alert{reduction} technique
\item Exploit concurrency
\item Contextual unfoldings can be exponentially smaller than P/N unfoldings
\end{itemize}

\vspace{1em}
\centerline{\includegraphics{fig/exploit-concurrency.fig}}

\end{frame}

% *****************************************************************************
\begin{frame}{Motivation and Goals}

[1] Baldan et al. McMillan's complete prefix for contextual nets.  In
\emph{Transactions of PNOMC}, p. 199-220, Berlin, 2008. Springer-Verlag.

\vspace{.3cm}

\begin{block}{Motivation}
Exploit the concurrency of contextual unfoldings to produce smaller unfoldings.
\end{block}

\begin{block}{Goals}
\begin{itemize}
\item Implementation of a contextual net unfolder following the abstract
algorithm proposed in [1].
\item To pave the way between the theory in [1] and its practical application.
\end{itemize}
\end{block}

\end{frame}

% ****************************************************************************
\begin{frame}{Event Histories}
\begin{columns}[T]
\column{.6\textwidth}

\begin{block}{Definition}
A \alert{history for $e$} is any set $H \subseteq T'$ verifying: 

\begin{enumerate}
\item $e \in H$,
\item All events of $H$ can be arranged to form a \emph{run},
\item \emph{Any} run of the events of $H$ fires $e$ last.
\end{enumerate}
\end{block}

%\begin{block}{Definition}
%$H_1$, $H_2$ in \alert{conflict}, written $H_1 \confl H_2$ iff
%\begin{enumerate}
%\item $\forall e_1 \in H_1,\ \forall e_2 \in H_2 \setminus H_1,\ \lnot (e_2
%\nearrow e_1)$, or
%\item $\forall e_1 \in H_2,\ \forall e_2 \in H_1 \setminus H_2,\ \lnot (e_2
%\nearrow e_1)$
%\end{enumerate}
%\end{block}

\begin{itemize}
\item One history per event in P/N unfoldings
\item \alert{Enriched events}
\item Unfolding algorithm deals with $\enr N$
\end{itemize}

\column{.1\textwidth}
\column{.28\textwidth}
%\vspace{-.6cm}
\includegraphics{fig/event-histories.fig}

\end{columns}
\end{frame}

% ****************************************************************************
\begin{frame}{Contextual Unfolding Algorithm}

\begin{algorithmic}
\REQUIRE A 1-safe contextual net $N = \langle P, T, F, C, m_0 \rangle$
\ENSURE A complete enriched prefix $\enr N$ of the full unfolding of $N$

\vspace{1ex}
\STATE Append to $\enr N$ a copy of $m_0$

\STATE $E = \peupdate (\enr N)$
\WHILE {$E \not = \emptyset$}
\STATE Remove from $E$ some $(e, H)$ minimal w.r.t $\prec$
\IF{not $\iscutoff (e, H)$}
\STATE Append $(e, H)$ to $\enr N$
\STATE Append a copy of $\post{f_T (e)}$ to $\enr N$
\STATE $E = E \cup \peupdate (\enr N)$
\ENDIF
\ENDWHILE
\end{algorithmic}
\end{frame}

% ****************************************************************************
\begin{frame}
\centerline{\LARGE Contributions}
\end{frame}

% ****************************************************************************
\begin{frame}{Possible Extensions}

\begin{itemize}
\item Addition of pairs $(e, H)$ to $\enr N$ renders $E$ out of date
\begin{itemize}
\item New markings are reachable
\end{itemize}
\item Idea: compute sets of \alert{concurrent conditions} that includes $c$
\end{itemize}

\centerline{\includegraphics{fig/possible-extensions.fig}}

\begin{itemize}
\item Algorithm missing in [1]
\item Contribution: \alert{algorithm} to compute the possible extensions
\item Contribution: the \alert{graph $\hst N$}, storing the histories of $\enr
N$
\end{itemize}

\end{frame}

% ****************************************************************************
\begin{frame}{Adequate Orders and Completeness}
\begin{itemize}

%\item \only<1-2>{Every \alert{run} in $\enr N$ $\leadsto$ run in $N$}
%\only<2>{$\leadsto$ \alert{marking} in $N$} \only<3->{\alert{Run} in $\enr N$
%$\leadsto$ \alert{marking} in $N$}

\item \alert{Run} in $\enr N$ $\leadsto$ \alert{marking} in $N$

\item<2-> $\enr N$ is \alert{complete} iff every reachable marking of $N$ is
represented in $\enr N$

\item<3-> Idea:
\begin{enumerate}
\item Order $\prec$ on pairs $(e, H)$
\item Append extensions $(e, H)$ to $\enr N$ respecting $\prec$
\item Define certain pairs $(e, H)$ as \alert{cutoffs}
\end{enumerate}

\item<4-> For Petri Nets, \alert{adequate orders} can produce exponentially
smaller unfoldings

\item<5-> Contribution:
\begin{enumerate}
\item Generalization of adequate orders for contextual unfoldings
\item Proof of finiteness and completeness of $\enr N$
\end{enumerate}

\end{itemize}
\end{frame}

% ****************************************************************************
\begin{frame}{Implementation and Testing}
\begin{itemize}
\item 4700 lines of code in C
\item From scratch, modular
\end{itemize}

\begin{block}{Testing}
\begin{itemize}
\item Isomorphism for Petri Nets
\item Reachability set
\end{itemize}
\end{block}

\begin{block}{Some test cases}
\begin{itemize}
\item Results of 6 examples out of 36
\end{itemize}
\vspace{-2em}
\begin{center}
\rowcolors[]{1}{orange!25}{orange!15}
\begin{tabular}{|lllllll|}
Name        & \co{rw} & \co{mutual} & \co{elevator} & \co{ab\_gesc} &
		\co{sdl\_arq} & \co{buf100} \\
Transitions & 36 & 41 & 51 & 52 & 96 & 101 \\
\hline
%Events      & 129 & 307 & 293 & 471 & 197 & 5051 \\
Histories   & 147 & 497 & 293 & 471 & 199 & 5051 \\
Time (s)    & 67 & $< 1$ & $< 1$ & 156 & 137 & 161 \\
\end{tabular}
\end{center}
\end{block}

\end{frame}

% ****************************************************************************
\begin{frame}[t]{Optimization -- Binary Concurrency Relations}
\begin{block}{Fact}
Computation of \peupdate{} on $\enr N$ spends more than 90\% of the time.
\end{block}

\begin{itemize}
\item<2-> Computing possible extensions require computing concurrent sets
\item<3-> For Petri Nets, \alert{binary} concurrency relation $\mid$ on
conditions
\begin{itemize}
\item<3-> $\{c_1, \ldots, c_n\}$ concurrent iff $c_i \mid c_j$ for all $i < j$
\end{itemize}

\item<4-> In contextual nets, read arcs prevent binary concurrency relations.
\only<4>{\centerline{\includegraphics{fig/optimization.fig}}}

\item<5-> However we can define a binary relation on \alert{enriched conditions}:
\begin{itemize}
\item<5-> Pairs condition, history
\end{itemize}
\end{itemize}

\uncover<5>{
\begin{block}{Theorem 17}
\footnotesize
$\rho \parallel \rho' \iff \bigwedge_{\rho_i \in \pre{\pre\rho}}
\rho_i \parallel \rho' \; \land \; \bigwedge_{\sigma_j \in \cont{\pre\rho}}
\sigma_j \parallel \rho' \; \land \; (\rho' \notin \pre{\pre \rho}) \; \land \;
\lnot \exists e'' \in H' \setminus H,\, \cont{e''} \cap \pre e \not=
\emptyset$
\end{block}}

\end{frame}

% ****************************************************************************
\begin{frame}{Summary}
\begin{itemize}
\item Unfoldings: state space reduction technique
\item Contextual nets: Petri Nets extended with read arcs
\item Contextual unfoldings can be exponentially smaller
\end{itemize}

\begin{block}{Contributions}
\begin{itemize}
\item Unfolder
\item Concrete algorithms computing the \emph{possible extensions}
\item Data structures: $\hst N$ graph, direct asymmetric conflict
\item Integration of \emph{adequate orders}
\item Binary \emph{concurrency relation}
\end{itemize}
\end{block}

\begin{block}{Future work}
\begin{itemize}
\item PE optimization: concurrency relations
\item Application
\end{itemize}
\end{block}

\end{frame}

% ****************************************************************************
\begin{frame}
%\thispagestyle{empty}
\titlepage
\begin{center}
\textit{Thank you for your attention. Questions?}
\end{center}
\end{frame}

%\begin{frame}
%\begin{center}
%\LARGE Appendix
%\end{center}
%\end{frame}
%
%\begin{frame} \frametitle{References}
%\linespread{0.95}
%\nocite{*}
%\bibliographystyle{plain}
%\bibliography{biblio}
%%\linespread{1.3}
%\end{frame}

\end{document}

